\pagebreak\section*{May 6, 2021}

We'll continue discussing properties of the spectrum of a bounded linear operator today: recall that the \textbf{resolvent} of an operator $A$ is the set of complex numbers $\lambda$ such that $A - \lambda$ is an element of $GL(H)$ (in other words, $A - \lambda$ is bijective, meaning it has a bounded inverse), and the \textbf{spectrum} of $A$ is the complement of the resolvent in $\CC$. While the spectrum is just the set of eigenvalues for matrices in a finite-dimensional vector space, there's a more subtle distinction to be made now: we define $\lambda \in \text{Spec}(A)$ to be an \textbf{eigenvalue} if there is some vector $u$ with $(A - \lambda)u = 0$, so $\lambda$ is in the spectrum because $A - \lambda$ is not injective. But there are other reasons for why $\lambda$ might be in the spectrum as well, for instance if the image is not closed. 

Last time, we proved that the spectrum is closed and is contained within the ball of radius $||A||$, meaning that it is compact. We then focused our attention on self-adjoint operators, and that's where we'll be directing our study today. We proved last lecture that a self-adjoint bounded linear operator $A$ always has $\langle Au, u \rangle$ real, and that it satisfies $||A|| = \sup_{||u|| = 1} |\langle Au, u \rangle|$. Here's our next result:

\begin{theorem}\label{spectrumisinrealline}
Let $A = A^\ast \in \mc{B}(H)$ be a self-adjoint operator. Then the spectrum $\text{Spec}(A) \subset [-||A||, ||A||]$ is contained within a line segment on the real line, and at least one of $\pm ||A||$ is in $\text{Spec}(A)$.
\end{theorem}

\begin{proof}
First, we'll show the first property (that the spectrum is contained within this given line segment). We know from last time that $\text{Spec}(A) \subset \{|\lambda| \le ||A||\}$, so we just need to show that $\text{Spec}(A) \subset \RR$ (in other words, any complex number with a nonzero imaginary part is in the resolvent). Write $A = s + it$ for $s, t$ real and $t \ne 0$, so that 
\[
    A - \lambda = (A - s) - it = \tilde{A} - it,
\]
where $\tilde{A} = A-s$ is another self-adjoint bounded linear operator because $(A-sI)^\ast = A^\ast - (sI)^\ast = A - sI$. So it suffices to show that $\tilde{A} - it$ is \textbf{bijective}, and we'll switch our notation back to using $A$ instead of $\tilde{A}$. 

Note that because $\langle Au, u \rangle$ is real, 
\[
    \text{Im}(\langle(A-it)u, u \rangle) = \text{Im}(\langle-itu, u \rangle) = -t ||u||^2,
\]
so $(A - it)u = 0$ only if $u = 0$ (since that's the only instance where the right-hand side is zero). Therefore, $A-it$ is injective, and we just need to show that it is surjective. Notice that $(A - it)^\ast = A+it$ is also injective by the same argument, so 
\[
    \text{Range}(A-it)^\perp = \text{Null}((A-it)^\ast) = \{0\}.
\]
And now we can use what we know about orthogonal complements: 
\[
    \overline{\text{Range}(A-it)} = (\text{Range}(A-it)^\perp)^\perp = \{0\}^\perp = H,
\]
so it suffices to show that the range of $A-it$ is closed. To show that, suppose we have a sequence of elements $u_n$ such that $(A-it)u_n \to v$; we want to show that $v \in \text{Range}(A-it)$. We know from the calculation above that
\[
    |t| \cdot ||u_n - u_m||^2 = |\text{Im}(\langle(A-it)(u_n - u_m), u_n-u_m\rangle)| \le |\langle(A-it)(u_n - u_m), u_n-u_m\rangle|,
\]
and by Cauchy-Schwarz this is bounded by
\[
    \le ||(A-it)u_n - (A-it)u_m|| \cdot ||u_n-u_m||.
\]
Simplifying the first and last expressions, we find that
\[
    ||u_n - u_m|| \le \frac{1}{|t|}||(A-it)u_n - (A-it)u_m||.
\]
Since $t$ is a fixed constant, and our sequence $\{(A-it)u_n\}$ converges, it is also Cauchy. In particular, for any $\eps > 0$, we can find some $N$ so that the right-hand side is smaller than $\eps$ as long as $n, m \ge N$, and that same $N$ shows that our sequence $\{u_n\}$ is also Cauchy. Therefore, there exists some $u \in H$ so that $u_n \to u$ by completeness of our Hilbert space, and now we're done: since $(A-it)$ is a bounded and thus continuous linear operator, 
\[
    (A-it)u = \lim_{n \to \infty} (A-it)u_n = v.
\]
So the range is closed, and combining this with our previous work, $A-it$ is surjective. This finishes our proof that $A-it$ is bijective and thus complex numbers with nonzero imaginary part are in the resolvent.

Now for the second property, since we have shown that $||A|| = \sup_{||u|| = 1} |\langle Au, u \rangle|$, there must be a sequence of unit vectors $\{u_n\}$ such that $|\langle Au_n, u_n \rangle| \to ||A||$. Since each term in this sequence is real, there must be a subsequence of these $\{u_n\}$ with $\langle Au_n, u_n\rangle$ converging to $||A||$ or to $-||A||$, which means that we have 
\[
    \langle (A \mp ||A||)u_n, u_n \rangle \to 0
\]  
as $n \to \infty$ (this notation means \textbf{one of} $-$ or $+$, depending on whether we had convergence to $||A||$ or $-||A||$). We claim that this means $A \mp ||A||$ is not invertible: assume for the sake of contradiction that it were invertible. Then
\[
    1 = ||u_n|| = ||(A \pm ||A||)^{-1} (A \mp ||A||) u_n|| \le ||(A \pm ||A||)^{-1}|| \cdot ||(A \mp ||A||) u_n||,
\]
but the right-hand side converges to $0$ as $n \to \infty$, contradiction. So $A \mp ||A||$ is not bijective, and thus one of $\pm ||A||$ must be in the spectrum of $A$, finishing the proof. 
\end{proof}

We can in fact strengthen this bound even more:

\begin{theorem}
If $A = A^\ast \in \mc{B}(H)$ is a self-adjoint bounded linear operator, and we define $a_- = \inf_{||u|| = 1} \langle Au, u \rangle$ and $a_+ = \sup_{||u|| = 1} \langle Au, u \rangle$, then $a_{\pm}$ are both contained in $\text{Spec}(A)$, which is contained within $[a_-, a_+]$. 
\end{theorem}
\begin{proof}
Applying a similar strategy as before, we know that because $-||A|| \le \langle Au, u \rangle \le ||A||$ for all $u$, we must have $-||A|| \le a_- \le a_+ \le ||A||$ (by taking the infimum and supremum of the middle quantity). Now by the definition of $a_-, a_+$, there exist \textbf{two sequences} $\{u_n^{\pm}\}$ of unit vectors so that $\langle Au_n^{\pm}, u_n^{\pm} \rangle \to a_{\pm}$. And the argument we just gave works here very similarly: since we know that
\[
    \langle (A - a_{\pm}) u_n^{\pm}, u_n^{\pm} \rangle \to 0,
\]
this implies that $a_+$ and $a_-$ are both in the spectrum because we have convergence to both points. 

It remains to show that the spectrum is contained within $[a_-, a_+]$. Let $b = \frac{a_- + a_+}{2}$ be their midpoint, and let $B = A - bI$. Since $b$ is a real number, $B$ is also a bounded self-adjoint operator, so by \cref{spectrumisinrealline}, we know that
\[
    \text{Spec}(B) \subset [-||B||, ||B||].
\]
This means that (shifting by $bI$)
\[
    \text{Spec}(A) \subset[-||B|| + b, ||B| + b],
\]
and we can finish by noticing that
\[
    ||B|| = \sup_{||u|| = 1} |\langle Bu, u \rangle| = \sup_{||u|| = 1} \left|\langle Au, u \rangle - \frac{a_+ + a_-}{2}\right|.
\]
Since $\langle Au, u \rangle$ always lies in the line segment $[a_-, a_+]$ (getting arbitrarily close to the endpoints), and $\frac{a_+ + a_-}{2}$ is their midpoint, this supremum will be half the length of that line segment, meaning that
\[
    ||B|| = \frac{a_+ - a_-}{2} \implies \text{Spec}(A) \subset[-||B|| + b, ||B| + b] = [a_-, a_+],
\]  
as desired, completing the proof.
\end{proof}

\begin{corollary}
Let $A^\ast = A \in \mc{B}(H)$ be a self-adjoint linear operator. Then $\langle Au, u \rangle \ge 0$ for all $u$ if and only if $\text{Spec}(A) \subset [0, \infty)$.
\end{corollary}

(This can be shown by basically walking through the logic for what $a_-$ needs to be under either of these conditions.)

We'll now move on to the spectral theory for self-adjoint \textbf{compact} operators: the short answer is that we essentially see just the eigenvalues, with the exception of zero being a possible accumulation point. And in particular, the spectrum will be countable, and this should make sense because compact operators are the limit of finite rank operators -- we don't expect to end up with wildly different behavior in the limit.

\begin{definition}
Let $A \in \mc{B}(H)$ be a bounded linear operator. We denote $E_\lambda$ to be the nullspace of $A - \lambda$, or equivalently the set of eigenvectors $\{u \in H: (A-\lambda)u = 0\}$.
\end{definition}

\begin{theorem}\label{selfadjointeigenspace}
Suppose $A^\ast = A \in \mc{B}(H)$ is a compact self-adjoint operator. Then we have the following:
\begin{enumerate}
\item If $\lambda \ne 0$ is an eigenvalue of $A$, then $\lambda \in \RR$ and $\dim E_\lambda$ is finite. 
\item If $\lambda_1 \ne \lambda_2$ are eigenvalues of $A$, then $E_{\lambda_1}$ and $E_{\lambda_2}$ are orthogonal to each other (every element in $E_{\lambda_1}$ is orthogonal to every element in $E_{\lambda_2}$). 
\item The set of nonzero eigenvalues of $A$ is either finite or countably infinite, and if it is countably infinite and given by a sequence $\{\lambda_n\}_n$, then $|\lambda_n| \to 0$.
\end{enumerate}
\end{theorem}
\begin{proof}
For (1), let $\lambda$ be a nonzero eigenvalue. Suppose for the sake of contradiction that $E_\lambda$ is infinite-dimensional. Then by the Gram-Schmidt process, there exists a countable collection $\{u_n\}_n$ of orthonormal elements of $E_{\lambda}$. Since $A$ is a compact operator, this means that $\{Au_n\}_n$ must have a convergent subsequence, and in particular that means we have a Cauchy sequence $\{Au_{n_j}\}_j$. But we can calculate
\[
    ||Au_{n_j} - Au_{n_k}||^2 = ||\lambda u_{n_j} - \lambda u_{n_k}||^2 = |\lambda|^2 ||u_{n_j} - u_{n_k}||^2 = 2|\lambda|^2,
\]
so the distance between elements of the sequence does not go to $0$ for large $j, k$, a contradiction. Thus $E_\lambda$ is finite-dimensional. To show that $\lambda$ must be real, notice that we can pick a unit-length eigenvector $u$ satisfying $Au = \lambda u$, and then we have
\[
    \lambda = \lambda \langle u, u \rangle = \langle \lambda u, u \rangle = \langle Au, u \rangle,
\]
and we've already shown that this last inner product must be real, so $\lambda$ is real.

For (2), suppose $\lambda_1 \ne \lambda_2$, and supppose $u_1 \in E_{\lambda_1}, u_2 \in E_{\lambda_2}$. Then 
\[
    \lambda_1 \langle u_1, u_2 \rangle = \langle \lambda_1 u_1, u_2 \rangle = \langle Au_1, u_2 \rangle,
\]
and now because $A$ is self-adjoint, this is 
\[
    = \langle u_1, Au_2 \rangle = \langle u_1, \lambda_2 u_2 \rangle = \lambda_2 \langle u_1, u_2 \rangle
\]
(no complex conjugate because eigenvalues are real). Therefore, we must have $(\lambda_1 - \lambda_2) \langle u_1, u_2 \rangle = 0$, so (because $\lambda_1 - \lambda_2 \ne 0$) $\langle u_1, u_2 \rangle = 0$ and we've shown the desired orthogonality. 

Finally, for (3), let $\Lambda = \{\lambda \ne 0: \lambda \text{ eigenvalue of }A\}$. We need to show that $\Lambda$ is either finite or countably infinite, and we claim that we can actually prove both parts of (3) simultaneously by showing that if $\{\lambda_n\}_n$ is a sequence of distinct eigenvalues of $A$, then $\lambda_n \to 0$. This is because the set
\[
    \Lambda_N = \{\lambda \in \Lambda: |\lambda| \ge \frac{1}{N}\}
\]
is a finite set for each $N$ (otherwise we could take any sequence of distinct elements in $\Lambda_N$, and that can't converge to $0$), and thus $\Lambda = \bigcup_{N \in \NN} \Lambda_N$ is a countable union of finite sets and thus countable. 

In order to prove this claim, let $\{u_n\}_n$ be the associated unit-length eigenvectors of our eigenvalues $\lambda_n$. Then 
\[
    |\lambda_n| = ||\lambda_n u_n|| = ||Au_n||,
\]
so we further reduce the problem to showing that $||Au_n|| \to 0$. But showing this is a consequence of us having an orthonormal sequence of vectors and $A$ being compact: suppose that $||Au_n||$ does not converge to $0$. Then there exists some $\eps_0 > 0$ and a subsequence $\{Au_{n_j}\}$ so that for all $j$, $||Au_{n_j}|| \ge \eps_0$. Then because $A$ is a compact operator, there exists a further convergent subsequence $e_k = u_{n_{j_k}}$, meaning that $\{Ae_k\}_k$ converges in $H$. 

Since $e_k$ and $e_\ell$ are eigenvectors that correspond to distinct eigenvalues, they are orthogonal, and therefore $Ae_k$ and $Ae_\ell$ are also orthogonal. But now if $f = \lim_{k \to \infty} Ae_k$, then 
\[
    ||f|| = \lim_{k \to \infty} ||Ae_k|| \ge \eps_0,
\]
meaning that by continuity of the inner product,
\[
    \eps_0^2 \le ||f||^2 = \langle f, f \rangle = \lim_{k \to \infty} \langle Ae_k, f \rangle = \lim_{k \to \infty} \langle e_k, Af \rangle.
\]
And because the sequence $\langle e_k, Af \rangle$ gives us the Fourier coefficients of $Af$, the sum of their squares should be finite (by Bessel's inequality, it's at most $||Af||^2 < \infty$). This contradicts the fact that the limit of the Fourier coefficients is at least $\eps_0^2$. So our original assumption is wrong, and $||Au_n||$ must converge to $0$, proving the claim.
\end{proof}