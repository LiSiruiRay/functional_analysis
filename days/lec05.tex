\pagebreak\section*{March 2, 2021}

We'll prove the Hahn-Banach theorem today, which explains how to extend bounded linear functionals on a subspace to the whole normed vector space, answering the question of whether the dual of bounded linear functionals is nontrivial for normed vector spaces. 

Last time, we discussed \textbf{Zorn's lemma} from set theory (which we can take as an axiom), which tells us that a partially ordered set has a maximal element if every chain has an upper bound. (Remember that this notion involves a generalization $\preceq$ of the usual $\le$.) As a warmup, today we'll use this axiom to prove a fact about vector spaces. Recall that a \vocab{Hamel basis} of a vector space $V$ is a linearly independent set $H$, where every element of $V$ is a finite linear combination of elements of $H$. We know that finite-dimensional vector spaces always have a (regular) basis, and this is the analog for infinite-dimensional spaces:

\begin{theorem}
If $V$ is a vector space, then it has a Hamel basis.
\end{theorem}
\begin{proof}
We'll construct a partially ordered set as follows: let $E$ be the set of linearly independent subsets of $V$, and we define a partial order $\preceq$ by inclusion of those subsets. We now want to apply Zorn's lemma on $E$, so first we must check the condition: if $C$ is a chain in $E$ (meaning any two elements can be compared), we can define 
\[
    c = \bigcup_{e \in C} e
\]
to be the union of all subsets in the chain. We claim that $c$ is a linearly independent subset: to see that, consider a subset of elements $v_1, v_2, \cdots, v_n \in c$. Pick $e_1, e_2, \cdots, e_n \in C$ such that $v_j \in e_j$ for each $j$: by induction, because we can compare any two elements in $C$, we can also order finitely many elements in $C$ as well, and thus there is some $J$ such that $e_j \preceq e_J$ for all $j \in [1, 2, \cdots, n]$. So that means that all of $v_1, \cdots, v_n$ are in $e_J$, which is a linearly independent set by assumption. So indeed our arbitrary set $v_1, \cdots, v_n \in c$ is linearly independent, meaning $c$ is linearly independent.

And now notice that $e \preceq c$ for all $e \in C$ -- that is, $c$ is an upper bound of $C$. So the hypothesis of Zorn is verified, and we can apply Zorn's lemma to see that $E$ has some maximal element $H$.

We claim that $H$ spans $V$ -- suppose otherwise. Then there is some $v \in V$ such that $v$ is not a finite linear combination of elements in $H$, meaning that $H \cup \{v\}$ is linearly independent. But then $H \prec H \cup\{v\}$ (meaning $\preceq$ but not equality), so $H$ is not maximal, which is a contradiction. Thus $H$ must have spanned $V$, and that means $H$ is a Hamel basis of $V$. 
\end{proof}

Now that we've seen Zorn's lemma in action once, we're ready to use it to prove Hahn-Banach:

\begin{theorem}[Hahn-Banach]\label{hahn-banach}
Let $V$ be a normed vector space, and let $M \subset V$ be a subspace. If $u: M \to \CC$ is a linear map such that $|u(t)| \le C||t||$ for all $t \in M$ (in other words, we have a bounded linear functional), then there exists a \textbf{continuous extension} $U: V \to \CC$ (which is an element of $\mc{B}(V, \CC) = V'$) such that $U|_M = u$ and $||U(t)|| \le C||t||$ for all $t \in V$ (with the same $C$ as above). 
\end{theorem}

This result is very useful -- in fact, it can be used to prove that the dual of $\ell^{\infty}$ is not $\ell^1$, even though the dual of $\ell^1$ is $\ell^{\infty}$.

To prove it, we'll first prove an intermediate result:

\begin{lemma}\label{hahnbanachlemma}
Let $V$ be a normed space, and let $M \subset V$ be a subspace. Let $u: M \to \CC$ be linear with $|u(t)| \le C||t||$ for all $t \in M$. If $x \not \in M$, then there exists a function $u': M' \to \CC$ which is linear on the space $M' = M + \CC x = \{t + ax: t \in M, a \in C\}$, with $u'|_M = u$ and  $|u'(t')| \le C |t'|$ for all $t' \in M'$.
\end{lemma}

We can think of $M$ as a plane and $x$ as a vector outside of that plane: then we're basically letting ourselves extend $u$ in one more dimension, and the resulting bounded linear functional has the same bound that $u$ did. The reason this is a helpful strategy is that we'll apply Zorn's lemma to the set of all continuous extensions of $u$, placing a partial order using extension. Then we'll end up with a maximal element, and we want to conclude that this maximal continuous extension is defined on $V$. So this lemma helps us do that last step of contradiction, much like with the proof of existence for a Hamel basis. 

Let's first prove the Hahn-Banach theorem assuming the lemma:

\begin{proof}[Proof of \cref{hahn-banach}]
Let $E$ be the set of all continuous extensions 
\[
    E = \{(v, N): N \text{ subspace of }V, M \subset N, v \text{ is a continuous extension of }u \text{ to }N\},
\]
meaning that it is a bounded linear functional on $N$ with the same bound $C$ as the original functional $u$. This is nonempty because it contains $(u, M)$. We now define a partial order on $E$ as follows:
\[
    (v_1, N_1) \preceq (v_2, N_2) \text{ if }N_1 \subset N_2, v_2|_{N_1} = v_1
\]
(in other words, $v_2$ is a continuous extension of $v_1$). We can check for ourselves that this is indeed a partial order, and we want to check the hypothesis for Zorn's lemma. To do this, let $C = \{(v_i, N_i): i \in I\}$ be a chain in $E$ indexed by the set $I$ (so that for all $i_1, i_2 \in I$, we have either $(v_{i_1}, N_{i_1}) \preceq (v_{i_2}, N_{i_2})$ or vice versa). 

So then if we let $N = \bigcup_{i \in I} N_i$ be the union of all such subspaces $N_i$, we can check that this is a subspace of $V$. This is not too hard to show: let $x_1, x_2 \in N$ and $a_1, a_2 \in \CC$. Then we can find indices $i_1, i_2$ such that $x_1 \in N_{i_1}$ and $x_2 \in N_{i_2}$, and one of these subspaces $N_{i_1}, N_{i_2}$ is contained in the other because $C$ is a chain. So (without loss of generality), we know that $x_1, x_2$ are both in $N_{i_2}$, and we can use closure in that subspace to show that $a_1x_1 + a_2x_2 \in N_{i_2} \subset N$.

And now that we have the subspace $N$, we need to make it into an element of $E$ by defining a linear functional $u: N \to \CC$ which satisfies the desired conditions. But the way we do this is not super surprising: we'll define $v: N \to \CC$ by saying that for any $t \in N$, we know that $t \in N_i$ for some $i$, and then we define $v(t) = v_i(t)$. But this is indeed well-defined: if $t \in N_{i_1} \cap N_{i_2}$, it is true that $v_{i_1}(t) = v_{i_2}(t)$, because we're still in a chain and thus one of $(v_{i_1}, N_{i_1})$ and $(v_{i_2}, N_{i_2})$ is an extension of the other by definition. Similar arguments (exercise to write out the details) also show that $v$ is linear, and that it's an extension of any $v_i$ (including the bound with the constant $C$). So $(v_i, N_i) \preceq (v, N)$, and we have an upper bound for our chain.

This means we've verified the Zorn's lemma condition, and now we can say that $E$ has a maximal element $(U, N)$. We want to show that $N = V$ (which would give us the desired conclusion); suppose not. Then there is some $x \in V$ that is not in $N$, and then \cref{hahnbanachlemma} tells us that there is a continuous extension $v$ of $U$ to $N + \CC x$, which must then also be a continuous extension of $u$. So $(v, N + \CC x)$ is an element of $E$, but that means $(U, N) \prec (v, N + \CC x)$, contradicting $(U, N)$ being a maximal element. So $N = V$ and we're done. 
\end{proof}

We'll now return to the (more computational) proof of the lemma:

\begin{proof}[Proof of \cref{hahnbanachlemma}]
We can check on our own that $M' = M + \CC x$ is a subspace (this is not hard to do), but additionally, we can show that the representation of an arbitrary $t' \in M'$ as $t + ax$ (for $t \in M$ and $a \in \CC$) is unique. This is because 
\[
    t + ax = \tilde{t} + \tilde{a}x \implies (a - \tilde{a})x = \tilde{t} -t \in M,
\]
which means that $x \in M$ (contradiction) unless $a = \tilde{a}$, which then implies that $t = \tilde{t}$. We need this fact because we want to define our continuous extension in a well-defined way: if we choose an arbitrary $\lambda \in \CC$, then the map 
\[
    u'(t + ax) = u(t) + a\lambda
\]
is indeed well-defined on $M'$, and then the map $u': M' \to \CC$ is linear. If the bounding constant $C$ is zero, then our map is just zero and we can extend that map by just using the zero function on $M'$. Otherwise, we can divide by $C$ and thus assume (without loss of generality) that $C = 1$. It remains to choose our $\lambda$ so that for all $t \in M$ and $a \in \CC$, we have $|u(t) + a\lambda| \le ||t + ax||$, which would show the desired bound and give us the continuous extension.

To do this, note that the inequality already holds whenever $a = 0$ (because it holds on $M$), so we just need to choose $\lambda$ to make the inequality work for $a \ne 0$. Dividing both sides by $|a|$ yields (for all $a \ne 0$)
\[
    \left|u\left(\frac{t}{-a}\right) - \lambda\right| \le \left|\left|\frac{t}{-a} - x\right|\right|.
\]
We know that $\frac{t}{-a} \in M$ because $t \in M$, so this bound is equivalent to showing that 
\[
    |u(t) - \lambda| \le |t - x| \quad \forall t \in M.
\]
To do this, we'll choose the real and imaginary parts of $\lambda$. First, we show there is some $\alpha \in \RR$ such that 
\[
    |w(t) - \alpha| \le ||t-x||
\]
for all $t \in M$, where $w(t) = \frac{u(t) + \overline{u(t)}}{2}$ is the real part of $u(t)$.  Notice that $|w(t)| = |\text{Re }u(t)| \le |u(t)| \le ||t||$ by assumption, and because $w$ is real-valued, 
\[
    w(t_1) - w(t_2) = w(t_1 - t_2) \le |w(t_1 - t_2)| \le ||t_1 - t_2||
\]
(the middle step here is where we use that $w$ is real-valued). Connecting this back to the expression $||t-x||$, we can add and subtract $x$ from above and use the triangle inequality to get
\[
    w(t_1) - w(t_2) \le ||t_1 - x|| + ||t_2 - x||.
\]
Thus, for all $t_1, t_2 \in M$, we have
\[
    w(t_1) - ||t_1 - x|| \le w(t_2) + ||t_2 - x||,
\]
and thus we can take the supremum of the left-hand side over all $t_1$s to get 
\[
    \sup_{t \in M} w(t) - ||t-x|| \le w(t_2) + ||t_2 - x||
\]
for all $t_2 \in M$, and thus
\[
    \sup_{t \in M} w(t) - ||t-x|| \le \inf_{t \in M} w(t) + ||t - x||.
\]
So \textbf{now we choose $\alpha$} to be a real number between the left-hand side and right-hand side, and we claim this value works. For all $t \in M$, we have
\[
    w(t) - ||t-x|| \le \alpha \le w(t) + ||t-x||,
\]
and now rearranging yields
\[
    -||t-x|| \le \alpha - w(t) \le ||t-x|| \implies |w(t) - \alpha| \le ||t-x||,
\]
and we've shown the desired bound. So now we just need to do something similar for the imaginary part, and we do so by repeating this argument with $ix$ instead of $x$. This then defines our function $u'$ on all of $M + \CC x$, and we're done (we can check that because the desired bound holds on both the real and imaginary ``axes'' of $x$, it holds for all complex multiples of $x$). 
\end{proof}